% !TEX TS-program = xelatex
% !TEX encoding = UTF-8 Unicode
% !Mode:: "TeX:UTF-8"

\documentclass{resume}
\usepackage{zh_CN-Adobefonts_external} % Simplified Chinese Support using external fonts (./fonts/zh_CN-Adobe/)
% \usepackage{NotoSansSC_external}
% \usepackage{NotoSerifCJKsc_external}
% \usepackage{zh_CN-Adobefonts_internal} % Simplified Chinese Support using system fonts
\usepackage{linespacing_fix} % disable extra space before next section
\usepackage{cite}
\usepackage{tabu}
\usepackage{progressbar}
\usepackage{multirow}

\begin{document}
\pagenumbering{gobble} % suppress displaying page number
\Large{
  \begin{tabu}{ c l r }
    \multirow{4}{1in}{\includegraphics[width=1in]{profile}} & \scshape{高亚虎} & {Linux~}\progressbar{0.650} \\
    & \email{gao\_yahu@163.com} & {Python~}\progressbar{0.60} \\
    & \phone{(+86) 131-2029-0626\ \ \ \ } & {C~}\progressbar{0.60} \\
    & \github[github.com/yahugao]{https://github.com/yahugao} & {shell~}\progressbar{0.3}
  \end{tabu}
}
\section{\faGraduationCap\ 知识与技能}\normalsize
\begin{itemize}
  \item {对Linux内核代码较为熟悉,对kernel的中断管理、内存管理和进程管理有过较为深入的了解。}
  \item {深入了解过Linux系统发行版构建工具Yocto,能根据需要对Recipe和Meta进行设计和调试。}
  \item {熟悉虚拟化工具QEMU, 并围绕QEMU对动态二进制翻译技术有过较深的研究。}
  \item {对Linux内核漏洞的发布流程非常熟悉,能够根据披露的漏洞信息快速查找相应的补丁。}
  \item {能熟练使用调试器GDB; 版本控制器Git; 文本编辑器Vim和Emacs; 熟悉rpmbuild工具。}
\end{itemize}

\section{\faUsers\ 工作经历}\normalsize
\datedsubsection{\textbf{风河系统软件有限公司} (北京)}{2017年9月 -- 2019年11月,2020年11月-今}
\role{系统工程师}{职责: windriver linux的维护和升级(人事外包)}
\begin{itemize}
  \item {主要工作内容: 负责WindRiver Linux系统的维护与升级。调查和修复客户系统中出现的各种bug,并对客户做出解释。对NVD网站公布的软件漏洞进行监控,修复对系统有较大影响的CVE漏洞。按照客户的要求对系统做出定制化的改动。}
\item {主要业绩:在问题可能无法复现的前提下,修复内核中的空指针异常、proc文件系统中潜在的use after free、系统启动过程中的watchdog重启、OOM触发的Panic导致watchdog重启、 调查系统内存充裕时的OOM等问题。对系统异常现象进行调查分析,并提出解决方案。利用Gitolite, git-daemon, cgit,Apache2和grokmirror等工具为团队搭建了git仓库管理系统。}
  \end{itemize}

% \end{itemize}

\section{\faUsers\ 实习经历}\normalsize
\datedsubsection{\textbf{英特尔(中国)有限公司} (北京)}{2016年8月 -- 2017年2月}
\role{实习软件工程师}{职责: LTP(Linux test project)由PC端向移动端的移植与验证}
\begin{itemize}
  \item {主要工作内容:将Linux测试工具LTP移植到基于X86平台的车载设备Brillo上。}
    \item {主要收获:完成了LTP由PC向车载Android平台Brillo的移植,解决了LTP在平台迁移过程中涉及的不兼容问题。成功向LTP开源社区提交了部分patch,成为LTP开源项目的contributer.}
\end{itemize}

\section{\faGraduationCap\  教育背景}\normalsize
\datedsubsection{\textbf{中国人民解放军信息工程大学} (河南·郑州)}{2014年9月 -- 2017年7月}
\role{硕士研究生}{专业:计算机科学与技术 方向:软件分析与逆向工程}
\begin{itemize}
\item {主要成就:设计并实现了在动态二进制翻译过程中,对本地库函数的调用,极大的提升了动态二进制翻译系统的效率。实现了C程序利用静态翻译生成的指令在目标平台的封装。}
  \item {期间发表学术论文《动态二进制翻译中的API优化》、《一种基于二进制翻译的共享库文件移植技术》, 学位论文《面向动态链接库的二进制翻译技术研究》。}
\end{itemize}
\datedsubsection{\textbf{河北理工大学} (河北·唐山)}{2010年9月 -- 2014年7月}
\role{学士}{专业:计算机科学与技术}
%\datedsubsection{\textbf{\LaTeX\ 简历模板}}{2015 年5月 -- 至今}
%\role{\LaTeX, Python}{个人项目}
%\begin{onehalfspacing}
%优雅的 \LaTeX\ 简历模板, https://github.com/billryan/resume
%\begin{itemize}
%  \item 容易定制和扩展
%  \item 完善的 Unicode 字体支持,使用 \XeLaTeX\ 编译
%  \item 支持 FontAwesome 4.5.0
%\end{itemize}
%\end{onehalfspacing}

% Reference Test
%\datedsubsection{\textbf{Paper Title\cite{zaharia2012resilient}}}{May. 2015}
%An xxx optimized for xxx\cite{verma2015large}
%\begin{itemize}
%  \item main contribution
%\end{itemize}


% \section{\faCogs\ IT 技能}\normalsize
% increase linespacing [parsep=0.5ex]
%\begin{itemize}[parsep=0.5ex]
%  \item {编程语言: C == Python > Java > C++}
%  \item 平台: Linux
%  \item 工具: Yocto,rpm,vim
%\end{itemize}
%\datedsubsection{\textbf{COURSERA 认证}}{2018年10月}
%\begin{itemize}[parsep=0.5ex]
%	\item 证书名称: Machine Learning
%	\item 证书编号: NFCZEBTZ5T4L
%	\item 证书URL : https://www.coursera.org/account/accomplishments/verify/NFCZEBTZ5T4L
%\end{itemize}


%\section{\faHeartO\ 获奖情况}
%\datedline{\textit{第一名}, xxx 比赛}{2013 年6 月}
%\datedline{其他奖项}{2015}

\section{\faInfo\ 其他}\normalsize
% increase linespacing [parsep=0.5ex]
\begin{itemize}
\item {多年的debug经验,工作认真,思维严谨。}
\item {喜欢健身、跑步,身体健康。}
\item {自学机器学习、深度学习和自然语言处理等技术,获得了coursera机器学习证书} \href{https://www.coursera.org/account/accomplishments/verify/NFCZEBTZ5T4L}{coursera verify}{,具有良好的学习能力和自我驱动能力。}
\end{itemize}

%% Reference
%\newpage
%\bibliographystyle{IEEETran}
%\bibliography{mycite}
\end{document}

%%% Local Variables:
%%% mode: latex
%%% TeX-master: t
%%% End:
