% !TEX TS-program = xelatex
% !TEX encoding = UTF-8 Unicode
% !Mode:: "TeX:UTF-8"

\documentclass{resume}
\usepackage{zh_CN-Adobefonts_external} % Simplified Chinese Support using external fonts (./fonts/zh_CN-Adobe/)
% \usepackage{NotoSansSC_external}
% \usepackage{NotoSerifCJKsc_external}
% \usepackage{zh_CN-Adobefonts_internal} % Simplified Chinese Support using system fonts
\usepackage{linespacing_fix} % disable extra space before next section
\usepackage{cite}

\begin{document}
\pagenumbering{gobble} % suppress displaying page number

\rightline{\email{gao\_yahu@163.com}}
\rightline{\phone{(+86) 131-2029-0626}}
\rightline{\github[Github主页]{https://github.com/YahuGao}}

\name{\leftline{高亚虎}}

\section{\faCogs\ 知识与技能}
\begin{itemize}[parsep=0.5ex]
  \item {熟悉文本表征BOW和TF-IDF, 词表征one-hot和word2vec, 了解ELMo和Bert}
  \item {熟悉LSTM, 了解传统的机器学习算法LR、SVM、DT和神经网络模型CNN、RNN、RCNN、FastText、HAN}
  \item {了解并具有keras和Pytorch的使用经验; 能熟练使用Emacs、Vim和Git; 熟悉Linux系统, 了解Linux系统的基本结构和文件系统。}
\end{itemize}

\section{\faCogs\ 项目经验}
\datedsubsection{\textbf{新闻文本分类}, 阿里云天池竞赛平台}{2020年07月15日 -- 2020年09月08日}
\begin{itemize}
  \item{训练集为对按照字符级别进行匿名处理的新闻数据,共有14个类别,20W样本,测试集为2W条样本。分别使用SVM、Naive Bayes、FastText、CNN、RNN、RCNN和HAN对训练集进行了拟合。以F1-score为衡量标准,SVM的效果最好达到了0.924, 其次为HAN得分为0.896. 对HAN模型进行了添加卷积层、使用动态学习率、早停、改变词嵌入维度、向其中的LSTM添加dropout并改变输出维度的调整,以缓解模型过拟合。}
\end{itemize}

% \datedsubsection{\textbf{微博立场检测}, FlyAI算法大赛}{}
% \begin{itemize}
% \item{TODO}
% \end{itemize}

\section{\faUsers\ 工作经历}
\datedsubsection{\textbf{瞬联软件科技有限公司}}{2017年09月 -- 2019年11月}
\role{系统工程师}{职责: Windriver Linux系统的维护}
\begin{itemize}
\item {调查和修复WindRiver Linux系统在运行中出现的各种bug, 在bug无法复现时结合堆栈信息和源码定位并修复问题; 对系统中涉及的CVE漏洞进行监控和修复。曾根据堆栈信息, 修复了三周复现一次的文件系统bug。}
\end{itemize}
% \section{\faUsers\ 实习经历}
\datedsubsection{\textbf{英特尔(中国)有限公司}}{2016年8月 -- 2017年2月}
\role{软件工程师(实习)}{职责: LTP(Linux test project)由PC端向移动端的移植与验证}
\begin{itemize}
\item {主要收获: 完成了LTP由PC向车载Android平台Brillo的移植,解决了LTP在平台迁移过程中的不兼容问题, 向LTP开源社区成功提交了修复数据类型的patch。}
\end{itemize}

\section{\faGraduationCap\  教育背景}
\datedsubsection{\textbf{中国人民解放军信息工程大学}}{2014年9月 -- 2017年7月}
\role{硕士研究生}{专业: 计算机科学与技术 方向: 软件分析与逆向工程}
\begin{itemize}
\item {主要成就: 设计并实现了在动态二进制翻译过程中, 对本地库函数的调用, 极大的提升了动态二进制翻译系统的效率; 实现了C程序利用静态翻译生成的指令在目标平台的封装。发表学术论文两篇(知网检索一篇), 学位论文一篇。}
\end{itemize}

\datedsubsection{\textbf{河北理工大学(现华北理工大学)}}{2010年9月 -- 2014年7月}
\role{学士}{专业: 计算机科学与技术}

\section{\faInfo\ 其他}
% increase linespacing [parsep=0.5ex]
\begin{itemize}[parsep=0.5ex]
  \item {语言: 英语 - 熟练(CET-4)}
  \item {爱好: 跑步,健身}
  \item {Gap说明: 自上一家公司离职后一直在国外陪读,8月份回国。}
\end{itemize}

\end{document}

%%% Local Variables:
%%% mode: latex
%%% TeX-master: t
%%% End:
